%!TEX program = xelatex
\documentclass[a4paper]{article}

% 1. Packages
\usepackage{amsmath, amssymb} % Standard math
%\usepackage{lmodern} % better rendering
\usepackage{multicol}
\usepackage{parskip}
\usepackage[protrusion=false]{microtype}
\emergencystretch 3em

% 2. Language Setup (Babel)
\usepackage[bidi=default, english,hebrew, provide=*]{babel}
\babeltags{english = english, hebrew = hebrew}

% 3. Fonts
% Hebrew Fonts (OpenType)
\babelfont[hebrew]{rm}[Renderer=Harfbuzz]{David CLM}
\babelfont[hebrew]{tt}{Miriam Mono CLM}
\babelfont[hebrew]{sf}{DejaVu Sans}


% 4. Macros
\newcommand{\LR}[1]{\textenglish{#1}}
\newcommand{\RL}[1]{\texthebrew{#1}} 
\newcommand{\EN}{\selectlanguage{english}}
\newcommand{\HE}{\selectlanguage{hebrew}}
\renewcommand{\labelitemi}{\LR{\textbullet}}


\begin{document}
	\title{מנוע \LR{XE\LaTeX}}
	\author{}
	\date{}
	\maketitle
	\section*{מה זה \LR{Xe\LaTeX}\ ?}
	מדובר במנוע קצת יותר מתקדם ממנוע ברירת המחדל \LR{(Pdf\LaTeX)} שמאפשר להשתמש בגופני \LR{\emph{unicode}} מהמערכת הפעלה. הוא עובד יותר טוב עם עברית מאשר ברירת המחדל, וברוב המקרים עדיף להשתמש בו.
	\subsection*{יתרונות}
	\begin{itemize}
		\item ניתן להשתמש בכל גופן שנמצא במערכת ההפעלה (כאן שומש David CLM) שהוא ברירת המחדל
	\item ניתן לשים מילים באנגלית בתוך עברית ללא סימון נוסף (רק מילה יחידה---לא
	 כמה)
	 \item ניתן לסמן טקסט מתוך ה--pdf בצורה נוחה (בניגוד ל-LuaLaTeX)
	\end{itemize}

 צורך ל-selection ישיר מה-pdf 
\subsubsection*{חסרונות}
\begin{itemize}
	\item לא עובד עם BiDi אוטמוטית (יש צורך לסמן אנגלית וכו למעט מילה יחידה)
	\item רשימות לא ממוספרות נראות מוזר (ניתן לראות כאן בדוגמא)
	\item זמן קומפילציה קצת ארוך יותר מ--pdfTeX
\end{itemize}
\end{document}

