\documentclass[a4]{article}
% Hebrew support with babel
\usepackage[english,hebrew]{babel}

\newcommand{\LR}[1]{\L{#1}}
\newcommand{\RL}[1]{\R{#1}}
\newcommand{\EN}{\selectlanguage{english}}
\newcommand{\HE}{\selectlanguage{hebrew}}
% = make tables work
\usepackage{array}
\newcolumntype{H}{>{\R\bgroup}r<{\egroup}} 

\title{מנוע ברירת המחדל \LR{Pdf\LaTeX}}
\author{}
\date{}

\begin{document}
	\maketitle
	\section*{מה זה מנוע ברירת המחדל?}
	
	זה המנוע שאלא אם כן מציינים אחרת, תוכנות \LR{\TeX} ועורכי קוד (כגון \LR{OverLeaf}) וכו משתמשים בו. הוא מנוע ישן שלא כלכך בנוי לעברית, אבל ניתן להשתמש בו עם עברית דיי בקלות ע"י שימוש ב--
	\LR{babel}. \textbf{דוגמא:}
\EN \verb|\usepackage[english,hebrew]{babel}|

\HE 
\subsection*{יתרונות}
\begin{itemize}
	\item עובד כמעט ללא \LR{Boiler-plate}
	\item זמן קופמפילציה מהיר במיוחד
	\item אין צורך במנוע מיוחד, אפשר להריץ ישר
\end{itemize}

\subsection*{חסרונות}
\begin{itemize}
	\item עבודה עם טבלאות לא פשוטה
	\item יש צורך לסמן כל טיפה של אנגלית בתוך עברית, גם אם זו מילה בודדת
	\item אין גמישות לגופנים, קשה לעשות \LR{Costimization}
\end{itemize}

	
	
\end{document}
