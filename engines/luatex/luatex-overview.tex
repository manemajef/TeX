%!TEX program = lualatex
\documentclass[a4paper]{article}

% 1. Packages
\usepackage{amsmath, amssymb} % Standard math
\usepackage{lmodern} % better rendering
\usepackage{multicol}
\usepackage{parskip}
\usepackage[protrusion=false]{microtype}
\emergencystretch 3em

% 2. Language Setup (Babel)
\usepackage[bidi=basic, layout=tabular]{babel}
\babelprovide[main, import=he, onchar=fonts ids]{hebrew}
\babelprovide[import=en, onchar=fonts ids]{english} 
\babeltags{english = english, hebrew = hebrew}

% 3. Fonts
% Hebrew Fonts (OpenType)
\babelfont[hebrew]{rm}[Renderer=Harfbuzz]{David CLM}
\babelfont[hebrew]{tt}{Miriam Mono CLM}
\babelfont[hebrew]{sf}{DejaVu Sans}


% 4. Macros
\newcommand{\LR}[1]{\textenglish{#1}}
\newcommand{\RL}[1]{\texthebrew{#1}} 
\newcommand{\EN}{\selectlanguage{english}}
\newcommand{\HE}{\selectlanguage{hebrew}}
\newcommand{\LC}{\char"200E\ }
\renewcommand{\labelitemi}{\LR{\textbullet}}


\begin{document}
	\title{מנוע Lua\LaTeX}
	\author{}
	\date{}
	\maketitle
\section*{מה זה LuaLaTeX?}
בגדול, זה המנוע הכי ״חכם״ ו-״מתקדם״. הייתרון העיקרי שלו הוא שהוא יודע לעבוד עם עברית ואנגלית באופן אותומטי ע״י שימוש ב-\texttt{bidi=basic}, ומוריד את הצורך לסמן אופן מפורש כל חלק באנגלית שמופיע בתוך פסקה עברית. 
\subsection*{יתרונות}
\begin{itemize}
	\item המנוע מפותח באופן פעיל, והוא המנוע המומלץ נכון לשנת 2026
	\item ניתן להשתמש בכל גופן שנמצא במערכת ההפעלה (כאן שומש David CLM) שהוא ברירת המחדל
	\item עבודה נוחה עם טבלאות ועמודות (ע״י \texttt{layout=tabular})
	\item \textbf{היתרון הבולט:}  עיבוד BiDi אוטומטי עם \texttt{bidi=basic}
\end{itemize}
	\subsubsection*{דוגמא לעיבוד \emph{BiDi} אוטומטי (השוואה עם pdf\TeX):}

	\textbf{עם  \texttt{bidi=basic}:}\\ \verb|כותבים עברית ואפשר to put English באמצע|
	
	\textbf{בלי \texttt{bidi=basic}:} \\ \texttt{בשביל לשים אנגלית בתוך עברית} \EN \verb|\LR{|\texttt{You must expliclty mark it}\verb|}| \HE \texttt{בצורה כזו}
\subsection*{חסרונות}
\begin{itemize}
\item  זמן קומפילציה ארוך יותר
\item בחירת טקסט מתוך ה-PDF לא עובדת כמו שצריך
\item יש יותר Boiler--plate 
\end{itemize}
\section*{מתי להשתמש ב-LuaTeX?}
\begin{itemize}
	\item כשהמסמך כולל הרבה החלפות מעברית לאנגלית
	\item כשהקומפילציה נעשית באופן מקומי (יכול להיות איטי בטירוף ב-OverLeaf)
	\item כשאין צורך ל-selection ישיר מה-pdf 
\end{itemize}
\end{document}

