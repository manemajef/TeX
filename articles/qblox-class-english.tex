\documentclass{article}

\usepackage{parskip}
\usepackage{graphicx}
\graphicspath{{media/}}
\usepackage{xcolor}
\usepackage[margin=100pt]{geometry}
\usepackage{tabularx}
\usepackage{paracol}

\begin{document}
	
	\title{Qblox at Home, in the Classroom, and in the Office}
	\author{Daniel Gat}
	\maketitle
	
	\section{A Glimpse into the World of Qblox Models}
	
	\textbf{The idea behind Qblox} is simple: to create a cube, sized comfortably for the palm of the hand, a cube capable of connecting in all three spatial directions to identical cubes — forward and backward, right and left, upward and downward. 
	
	Give the cubes the six colors of the rainbow along with black, gray, and white, and from these combinations of form and color, we begin to see what can be described as small works of art — suitable for decorating the home, the classroom, and the workplace.
	
	In the first image, one can see that the connection method is based on three channels arranged in rotational symmetry — somewhat reminiscent of the familiar organization of the blades of a wind turbine — yet unique in its ability to generate structures in three-dimensional space.
	
	\begin{center}
		\textbf{Symmetry in Space}
		
		Forward and backward, right and left, up and down
		
		\vspace{0.5em}
		\includegraphics{img-1.png}
	\end{center}
	
	\vspace{0.5em}
	
	\textbf{My Path of Exploration and Learning}
	
	In the remainder of this chapter, I will present to the reader my personal learning journey as I explored the creative possibilities and other pleasures of Qblox. To understand something of that path, we will begin by revealing my earliest exploratory steps, as well as more distant milestones. This should provide a sense of the direction of the process, the development of the results, and, to some extent, the working method I adopted.
	

	
	\subsection{The First Model: The Spinning Top in Cool Colors}
	
	Given the design of the core cube with its three connection channels, the first model I created — a four-cube spinning top — was hardly surprising. (Yes, it really spins easily — this will be shown later in a video.)
	
	\vspace{1em}
	\columnratio{0.6}
	
	\begin{paracol}{2}
		
		Already at this stage, I made use of the model beyond simply revealing its shape. I colored the four cubes in what artists and designers refer to as the group of \emph{cool colors}. Opposed to these is the group of \emph{warm colors}. We will return to the topic of color later.
		
		What is important here is that in the course of my exploration and learning, I realized how comfortable I felt working in a hierarchical, multi-layered manner. In other words, instead of building by adding a single cube at each step, I preferred to work with repeating components that themselves contain several cubes arranged in the same spatial organization.

		\switchcolumn
		
		\centering
		\includegraphics[width=0.8\linewidth]{img-2.png}
		
		\vspace{0.5em}
		Spinning Top
		
	\end{paracol}
			
	The present spinning top is one of the two principal components that will play a central role in my journey of exploration and learning.
	
	\vspace{1em}
	
	\columnratio{0.6}
	\begin{paracol}{2}
		
		\subsection{The Second Model: The Ring}
		
		Unlike the first component — the expected one, the spinning top — the second recurring component came as quite a surprise. It turns out that based on the three-channel cube, one can construct a minimal closed ring that always consists of ten cubes. As the image shows, the ring extends beyond a single plane.
		
		\switchcolumn
		
		\centering
		\includegraphics[width=0.8\linewidth]{img-3.png}
		
		\vspace{0.5em}
		The Ring
		
	\end{paracol}
	
	\subsection{Third Model: The Cocoon}
		
	\columnratio{0.3}
		Thanks to the unfamiliar geometry and symmetry of Qblox, the ring can connect further through its remaining open channels and transform into a more complex structure — the cocoon.
			
		Thanks to the unfamiliar geometry and symmetry of Qblox, the ring can connect further through its remaining open channels and transform into a more complex structure — the cocoon.
		
		In the world of planning and research, an unexpected discovery is sometimes a significant event. There is even a common term for it: \emph{serendipity}. We will return to this concept in another chapter, the one devoted to creativity.
		
\pagebreak

	\begin{paracol}{2}
		

	
	
	
	The process of transforming the ring into the cocoon is shown in the three-part image. In the upper illustration, I marked in red the channels that are meant to connect through the addition of four cubes. In the middle illustration, the four new cubes appear in yellow. Together with the six upper cubes, they form a new ring — and another hidden ring with the six lower cubes.
	
	

			\switchcolumn
	
	\centering
	\includegraphics[width=0.8\linewidth]{img-4.png}
	
	\vspace{0.5em}
	The Cocoon
	
		\end{paracol}
	\vspace{1em}



			The third, bottom illustration shows the second addition from the back, marked in orange. In total, eight cubes are added. The complete cocoon consists of eighteen cubes and five rings.
	
	
\end{document}