% !TeX program = lualatex

\documentclass{article}

\usepackage[hebrew, provide=*, bidi=basic, layout=tabular]{babel}
\babelprovide[import=en, onchar=fonts ids]{enlgish}
\babelfont[hebrew]{rm}{David CLM}
\babeltags{hebrew=hebrew, english=english}
\newcommand{\LR}[1]{\textenglish{#1}}
\newcommand{\RL}[1]{\texthebrew{#1}}
\renewcommand{\L}[1]{\LR{#1}}
\newcommand{\HE}{\selectlanguage{hebrew}}
\newcommand{\EN}{\selectlanguage{english}}

\usepackage{parskip}
\usepackage{graphicx}
\graphicspath{{media/}}
\usepackage{xcolor} % Added for the red frame check
\usepackage[margin=100pt]{geometry}
\usepackage{tabularx}
\usepackage{paracol}

\begin{document}
	\selectlanguage{hebrew}
	\title{קיובלוקס בבית, בכיתה ובמשרד}
	\author{דניאל גת}
	\maketitle
	
	\section{הצצה אל עולם דגמי קיובלוקס}
	\textbf{הרעיון מאחורי קיובלוקס}
	הוא פשוט: לחולל קובייה, בגודל הולם כף היד, קובייה שמסוגלת לחבור, בשלושת כיווני המרחב, אל קוביות זהות – קדימה ואחורה, ימין ושמאל, ומעלה מטה. להעניק לקוביות את ששת צבעי הקשת והשלישייה שחור, אפור ולבן, ועם אלה, לצרף צירופי צורה וצבע. ואז נראה מה שנראה במונחים של יצירות אומנות קטנות, לקישוט הבית, הכיתה ומקום העבודה. 
	
	בתמונה הראשונה רואים ששיטת החיבור היא באמצעות 3 תעלות המסודרות בסימטריה סיבובית – קצת ברוח ארגון הכנפיים המוכרות היטב של טורבינת רוח – אבל עם הייחודיות של יצירת מבנים במרחב.
	\begin{center}
		\textbf{סימטריה במרחב}
		
		לפנים ומאחור ימין ושמאל מעלה ומטה \\ 
		\vspace{0.5em}
		\LR{\includegraphics{img-1.png}}
	\end{center}
	\vspace{0.5em}
	
	\textbf{מסלול החיפוש והלימוד שלי}
	בהמשך פרק זה אציג לקורא את מסלול הלימוד שלי, בדרכי לברר את אפשרויות היצירה ושאר ההנאות של קיובלוקס. כדי להבין טיפה מאותו מסלול, נתחיל בחשיפה של צעדי הגישוש הראשונים שלי, וגם של אבני דרך רחוקות יותר. זה אמור לתת מושג על כיוון התהליך, על התפתחות התוצאות וגם, טיפה על שיטת העבודה שאימצתי.
	
	\pagebreak
	
	\subsection{הדגם הראשון: הסביבון עם הצבעים הקרים} 
	לאור העיצוב של קוביית המפתח עם 3 תעלות החיבור, כלל לא מפתיע הדגם הראשון שיצרתי, הסביבון בעל 4 הקוביות. (כן, הוא באמת מסתובב בקלות – זאת נראה בהמשך בסרטון). 
	
	\vspace{1em}
	\columnratio{0.6} 
	
	\begin{paracol}{2}

כבר כאן עשיתי בו שימוש נוסף מעבר לחשיפת הצורה. צבעתי את 4 הקוביות בצבעים שאומנים ומעצבים מכנים קבוצת הצבעים הקרים. מול אלה ישנה קבוצה של צבעים חמים. אבל בנושא הצבע נעסוק בהמשך. כאן חשוב לי לומר שבמסלול החיפוש/לימוד שלי ראיתי שנח לי מאד להשתמש בשיטת העבודה ההיררכית, הרב שכבתית. במילים אחרות, במקום לבנות בשיטת הוספה של קובייה בודדת בכל צעד, לעבוד עם רכיבים חזרתים אשר בעצמם מכילים מספר קוביות עם אותו ארגון מרחבי. והסביבון הנוכחי הוא אחד משני הרכיבים העיקריים שיככבו במסלול החיפוש והלימוד שלי. 
		\switchcolumn
		
		\selectlanguage{english}
		
		\includegraphics[width=0.8\linewidth]{img-2.png} \\
		\centering{\RL{הסביבון}}
	\end{paracol}

\selectlanguage{hebrew}

\vspace{1em}

\columnratio{0.6}
\begin{paracol}{2}
	\subsection{הדגם השני: הטבעת}	 
		בניגוד לרכיב הראשון, הצפוי, כלומר הסביבון, הרכיב החזרתי השני גרם לי להפתעה לא קטנה. מסתבר שעל בסיס קוביית 3 התעלות, ניתן לבנות טבעת סגורה מינימלית, אשר תמיד מכילה 10 קוביות. כפי שהתמונה מראה, הטבעת חורגת ממישור יחיד.
		\switchcolumn

		\EN
		\includegraphics[width=0.8\linewidth]{img-3.png} \\
		\centering{\RL{הטבעת}}
		

\end{paracol}
\HE
\subsection{דגם שלישי: הפקעת}
הודות לגיאומטריה ולסימטריה הלא מוכרות של קיובלוקס, הטבעת מאפשרת להתחבר, באמצעות תעלות פנויות בתוכה, ולשנות צורה לדגם מורכב יותר – הפקעת. בעולם התכנון והמחקר, תגלית לא צפויה היא לפעמים אירוע חשוב, ויש לו כינוי מקובל, סרנדיפיטי, שגם עליו נדבר בפרק אחר, זה שמוקדש ליצירתיות.

\vspace{1em}

\columnratio{0.6}
\begin{paracol}{2}
	תהליך הפיכת הטבעת לפקעת מוצג בתמונה המשולשת. באיור העליון סימנתי באדום את התעלות שאמורות להתחבר על ידי הוספת 4 קוביות. באיור האמצע רואים את ארבע הקוביות החדשות בצהוב. יחד עם השש העליונות נוצרת טבעת חדשה. ועוד טבעת, מוסתרת, עם השש התחתונות. האיור השלישי, התחתון, מראה את התוספת השנייה מאחור, מסומנת בצבע כתום. בסך הכול נוספות 8 קוביות, והפקעת כולה מכילה 18 קוביות וחמש טבעות.
	\switchcolumn
	
	\EN
	
	\includegraphics[width=0.9\linewidth]{img-4.png}
  \centering{\RL{הפקעת}}

\end{paracol}
\HE

	
\end{document}
